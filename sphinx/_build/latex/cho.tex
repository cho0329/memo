%% Generated by Sphinx.
\def\sphinxdocclass{ujbook}
\documentclass[a4paper,10pt,dvipdfmx]{sphinxmanual}
\ifdefined\pdfpxdimen
   \let\sphinxpxdimen\pdfpxdimen\else\newdimen\sphinxpxdimen
\fi \sphinxpxdimen=.75bp\relax
\ifdefined\pdfimageresolution
    \pdfimageresolution= \numexpr \dimexpr1in\relax/\sphinxpxdimen\relax
\fi
%% let collapsable pdf bookmarks panel have high depth per default
\PassOptionsToPackage{bookmarksdepth=5}{hyperref}

\PassOptionsToPackage{warn}{textcomp}


\usepackage{cmap}
\usepackage[T1]{fontenc}
\usepackage{amsmath,amssymb,amstext}







\usepackage{sphinx}

\fvset{fontsize=auto}
\usepackage[dvipdfm]{geometry}


% Include hyperref last.
\usepackage{hyperref}
% Fix anchor placement for figures with captions.
\usepackage{hypcap}% it must be loaded after hyperref.
% Set up styles of URL: it should be placed after hyperref.
\urlstyle{same}
\usepackage{pxjahyper}

\renewcommand{\contentsname}{Contents:}

\usepackage{sphinxmessages}
\setcounter{tocdepth}{1}


\renewcommand{\baselinestretch}{0.8}
\pagestyle{plain}
\thispagestyle{plain}
\setcounter{secnumdepth}{3}

\makeatletter
\renewcommand{\maketitle}{
  \ifsphinxpdfoutput
    \begingroup
      \def\\{, }
      \def\and{and }
      \pdfinfo{
        /Author (\@author)
        /Title (\@title)
      }
    \endgroup
  \fi
  \begin{center}
    \sphinxlogo%
    {\Large \@title} \par
  \end{center}
  \begin{flushright}
    \@date \hspace{3zw} \@author \par
    \py@authoraddress \par
  \end{flushright}
  \@thanks
  \setcounter{footnote}{0}
  \let\thanks\relax\let\maketitle\relax
}
\makeatother

\let\pyOldTableofcontents=\tableofcontents
\renewcommand{\tableofcontents}{
  \begingroup
  \parskip = 0mm
  \pyOldTableofcontents
  \endgroup
  \vspace{12pt}
}


\title{cho}
\date{2021年09月12日}
\release{}
\author{cho}
\newcommand{\sphinxlogo}{\vbox{}}
\renewcommand{\releasename}{}
\makeindex
\begin{document}

\pagestyle{empty}
\sphinxmaketitle
\pagestyle{plain}
\sphinxtableofcontents
\pagestyle{normal}
\phantomsection\label{\detokenize{index::doc}}



\chapter{src}
\label{\detokenize{modules:src}}\label{\detokenize{modules::doc}}

\section{camera package}
\label{\detokenize{camera:camera-package}}\label{\detokenize{camera::doc}}

\subsection{Submodules}
\label{\detokenize{camera:submodules}}

\subsection{camera.camera module}
\label{\detokenize{camera:module-camera.camera}}\label{\detokenize{camera:camera-camera-module}}\index{モジュール@\spxentry{モジュール}!camera.camera@\spxentry{camera.camera}}\index{camera.camera@\spxentry{camera.camera}!モジュール@\spxentry{モジュール}}
\sphinxAtStartPar
Sample機能提供モジュール
\index{Sample (camera.camera のクラス)@\spxentry{Sample}\spxextra{camera.camera のクラス}}

\begin{fulllineitems}
\phantomsection\label{\detokenize{camera:camera.camera.Sample}}\pysigline{\sphinxbfcode{\sphinxupquote{class }}\sphinxcode{\sphinxupquote{camera.camera.}}\sphinxbfcode{\sphinxupquote{Sample}}}
\sphinxAtStartPar
ベースクラス: \sphinxcode{\sphinxupquote{object}}

\sphinxAtStartPar
sample機能を実装したクラスです。
\index{add() (camera.camera.Sample のメソッド)@\spxentry{add()}\spxextra{camera.camera.Sample のメソッド}}

\begin{fulllineitems}
\phantomsection\label{\detokenize{camera:camera.camera.Sample.add}}\pysiglinewithargsret{\sphinxbfcode{\sphinxupquote{add}}}{\emph{\DUrole{n}{arg1}}, \emph{\DUrole{n}{arg2}}}{}
\sphinxAtStartPar
引数で指定した値を足し算して返します。\sphinxcode{\sphinxupquote{arg1 + arg2}}
\begin{quote}\begin{description}
\item[{パラメータ}] \leavevmode\begin{itemize}
\item {} 
\sphinxAtStartPar
\sphinxstyleliteralstrong{\sphinxupquote{arg1}} (\sphinxstyleliteralemphasis{\sphinxupquote{int}}) \sphinxhyphen{}\sphinxhyphen{} 足される値。

\item {} 
\sphinxAtStartPar
\sphinxstyleliteralstrong{\sphinxupquote{arg2}} (\sphinxstyleliteralemphasis{\sphinxupquote{int}}) \sphinxhyphen{}\sphinxhyphen{} 足す値。

\end{itemize}

\item[{戻り値の型}] \leavevmode
\sphinxAtStartPar
int

\item[{戻り値}] \leavevmode
\sphinxAtStartPar
足し算した結果。

\end{description}\end{quote}

\end{fulllineitems}

\index{bar (camera.camera.Sample の属性)@\spxentry{bar}\spxextra{camera.camera.Sample の属性}}

\begin{fulllineitems}
\phantomsection\label{\detokenize{camera:camera.camera.Sample.bar}}\pysigline{\sphinxbfcode{\sphinxupquote{bar}}\sphinxbfcode{\sphinxupquote{ = 1}}}
\sphinxAtStartPar
xxを保持するメンバです

\end{fulllineitems}

\index{foo (camera.camera.Sample の属性)@\spxentry{foo}\spxextra{camera.camera.Sample の属性}}

\begin{fulllineitems}
\phantomsection\label{\detokenize{camera:camera.camera.Sample.foo}}\pysigline{\sphinxbfcode{\sphinxupquote{foo}}\sphinxbfcode{\sphinxupquote{ = 1}}}
\sphinxAtStartPar
yyを保持するメンバです

\end{fulllineitems}


\end{fulllineitems}



\subsection{Module contents}
\label{\detokenize{camera:module-camera}}\label{\detokenize{camera:module-contents}}\index{モジュール@\spxentry{モジュール}!camera@\spxentry{camera}}\index{camera@\spxentry{camera}!モジュール@\spxentry{モジュール}}

\section{preprocessing module}
\label{\detokenize{preprocessing:module-preprocessing}}\label{\detokenize{preprocessing:preprocessing-module}}\label{\detokenize{preprocessing::doc}}\index{モジュール@\spxentry{モジュール}!preprocessing@\spxentry{preprocessing}}\index{preprocessing@\spxentry{preprocessing}!モジュール@\spxentry{モジュール}}\index{my\_sum() (preprocessing モジュール)@\spxentry{my\_sum()}\spxextra{preprocessing モジュール}}

\begin{fulllineitems}
\phantomsection\label{\detokenize{preprocessing:preprocessing.my_sum}}\pysiglinewithargsret{\sphinxcode{\sphinxupquote{preprocessing.}}\sphinxbfcode{\sphinxupquote{my\_sum}}}{\emph{\DUrole{n}{x}}, \emph{\DUrole{n}{y}}}{}
\sphinxAtStartPar
print:hahhaaS
\begin{description}
\item[{x}] \leavevmode{[}int{]}
\sphinxAtStartPar
operand1 of addition

\item[{y}] \leavevmode{[}int{]}
\sphinxAtStartPar
operand2 of addition

\end{description}
\begin{description}
\item[{int}] \leavevmode
\sphinxAtStartPar
sum of x and y

\end{description}

\end{fulllineitems}



\chapter{Indices and tables}
\label{\detokenize{index:indices-and-tables}}\begin{itemize}
\item {} 
\sphinxAtStartPar
\DUrole{xref,std,std-ref}{genindex}

\item {} 
\sphinxAtStartPar
\DUrole{xref,std,std-ref}{modindex}

\item {} 
\sphinxAtStartPar
\DUrole{xref,std,std-ref}{search}

\end{itemize}


\renewcommand{\indexname}{Pythonモジュール索引}
\begin{sphinxtheindex}
\let\bigletter\sphinxstyleindexlettergroup
\bigletter{c}
\item\relax\sphinxstyleindexentry{camera}\sphinxstyleindexpageref{camera:\detokenize{module-camera}}
\item\relax\sphinxstyleindexentry{camera.camera}\sphinxstyleindexpageref{camera:\detokenize{module-camera.camera}}
\indexspace
\bigletter{p}
\item\relax\sphinxstyleindexentry{preprocessing}\sphinxstyleindexpageref{preprocessing:\detokenize{module-preprocessing}}
\end{sphinxtheindex}

\renewcommand{\indexname}{索引}
\printindex
\end{document}